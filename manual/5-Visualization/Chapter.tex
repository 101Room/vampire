\chapter{Visualization}\label{chap:visualization}

\vampire provides tools for visualising systems using external programs such as Rasmol, Jmol and POV-Ray. To compile these utilities, use the following command in the main directory of your \vampire installation folder:

\begin{minipage}[c]{\textwidth}
\centering
\textit{make vdc}
\end{minipage}\\
 
 The \vampire data converter, or vdc, is run to produce the input files needed. By default, it provides output for both Rasmol and POV-Ray. 

%opt/vampire/cfg2xx\\

\section*{Getting started}
\addcontentsline{toc}{section}{Getting started}
To generate the positions of your atoms, multiple parameters must be set in the input file, depending on the number of cores the simulation is run on.  
min/max\\

\subsection*{input}
{\footnotesize
\begin{verbatim}
#------------------------------------------
# data output
#------------------------------------------
config:atoms
config:output-nodes = 12
config:output-rate     = 1000
config:output-format = binary
\end{verbatim}
}

add descriptors for each parameter\\

\section*{Atomic visualization with rasmol}
\addcontentsline{toc}{section}{Atomic visualization with rasmol}

To visualise your system using Rasmol, simply run vdc in the same directory as your output. The config:output files must be present.\\

This produced a file called crystal.xyz, which is a chemical file format with information on the atomic positions. The format of the .xyz format is as follows:\\

\subsection*{.xyz}
{\footnotesize
\begin{verbatim}
<number of atoms>
comment line
<element> <X> <Y> <Z>
...
\end{verbatim}
}

The element in the .xyz file does not necessarily need to be the same as the atoms used in your system. They can instead be chosen for a different colour palette depending on the users requirements.  

\section*{Atomic visualization with POV-Ray}
\addcontentsline{toc}{section}{Atomic visualization with POV-Ray}

To produce pictures of your material of punishable quality and high configurability, it is also possible to use POV-Ray. After running vdc, the file "spins.pov" contains all the necessary information and an image may be produced by using:

\begin{minipage}[c]{\textwidth}
\centering
\textit{povray spins.pov}
\end{minipage}\\

In a situation where there are multiple possible snapshots of the system, this command will render all snapshots in order. To select specific snapshots or ranges, you need to add the following flags:

\begin{minipage}[c]{\textwidth}
\centering
\textit{+KFF<N> (initial frame number)\\
+KFI<N> (final frame number)}
\end{minipage}\\

For example, to render frame 9 only, you could use:

\begin{minipage}[c]{\textwidth}
\centering
\textit{povray -W3600 -H2700 +A0.3 +KFI9 +KFF9 spins.pov}
\end{minipage}\\

Where the "-W" and "-H" flags define the width and heigh of the image (the resolution), and "+A" is used for antialiasing. \\ 

\subsection{Customisation options}
The POV-Ray output from vdc can be customised in several ways by using command line flags when running vdc. There are several choices of possible colourmap configurations, the ones provided by default are made to be perceptually uniform and in some cases take account of colourblindness.\\

\begin{minipage}[c]{\textwidth}
\centering
\textit{--colourmap [C2/BWR/Rainbow]}
\end{minipage}\\

The "C2" coloumap is cyclic and useful for 3D magnetic systems. It has four principle directions of magenta, yellow, green and blue. As it is cyclic, there will be a smooth transition between colour at all angles, irrespective of what is chosen as the $0^{\circ}$ spin direction.\\

The "BWR" colourmap can be applied to 2D systems where the focus is on opposite spin directions, such as antiferromagnetic or ferrimagentic systems. It offers two colours of contrast, red and blue, which are joined by a gradual fade to white. As it is not cyclic, care must be taken at the $0^{\circ}$ spin direction as moving from it to a negative angle gives a hard border of red to blue colour. \\

The "Rainbow" colourmap can be used in 3D systems where spins are aligned in many different directions such as vortex configurations. While it is still designed to be somewhat perceptually uniform, this is very difficult to do with rainbow palettes hence its use typically loses detail when compared to other maps, however it is also one of the most vibrant.

\begin{minipage}[c]{\textwidth}
\centering
\textit{vdc spins.pov --custom-colourmap file-name}
\end{minipage}\\

Finally, a user defined colourmap can be used. To apply a different map, a file containing 256 colours in the RBG format must be provided in the same directory that vdc is run.\\

POV-Ray images produced by vdc have a 3D brightening effect enabled by default. When spins do not lie only in the xy-plane, their colour brightness is increased or reduced depending on their magnitude in the z-axis. To disable this effect, the following flag can be used:

\begin{minipage}[c]{\textwidth}
\centering
\textit{vdc spins.pov --2D}
\end{minipage}\\

When used, this effect does not have to be applied in the z, or \{0,0,1\} direction.

\begin{minipage}[c]{\textwidth}
\centering
\textit{vdc spins.pov --vector-z \textbackslash\{x,y,z\textbackslash\}}
\end{minipage}\\

To define the axis, simply use the command "--vector-z" followed by a direction vector. This does not need to be normalised, and can be used without defining the xy-plane, as explained below.\\

Similarly, another direction can be defined for the $0^{\circ}$ direction in the xy-plane. However, as the z-axis effect is not necessarily in the \{0,0,1\} direction, this parameter cannot be used alone and must be accompanied by the flag "--vector-z".

\begin{minipage}[c]{\textwidth}
\centering
\textit{vdc spins.pov --vector-z \textbackslash\{x,y,z\textbackslash\} --vector-x \textbackslash\{x,y,z\textbackslash\}}
\end{minipage}\\

\section*{Micromagnetic visualization with PovRAY}
\addcontentsline{toc}{section}{Micromagnetic visualization with PovRAY}
cell2povray\\
macros\\
customization\\
colouring options\\

\section*{Visualization Movies}
\addcontentsline{toc}{section}{Visualization Movies}